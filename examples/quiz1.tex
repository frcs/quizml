%%
%% This document was generated by QuizML using the tcd-eleceng-latex.jinja template
%%
%% see www.github.com/frcs/quizml
%%
%%
\documentclass{tcdexams}




\def \infoinputbasename {quiz1}
  

\providecommand{\infoprogrammeyearname}{}
\providecommand{\infoexamsemester}{}
\providecommand{\infoexamyear}{}
\providecommand{\infoexamdate}{}
\providecommand{\infoexamtime}{}
\providecommand{\infoexamvenue}{}
\providecommand{\infomodulename}{}
\providecommand{\infomodulecode}{}
\providecommand{\infoexaminer}{}
\providecommand{\infoinstructions}{}
\providecommand{\infomaterials}{}
\providecommand{\infoadditionalinformation}{}

\programmeyearname{\infoprogrammeyearname}
\examsemester{\infoexamsemester}
\examyear{\infoexamyear}
\examdate{\infoexamdate}
\examtime{\infoexamtime}
\examvenue{\infoexamvenue}
\modulename{\infomodulename}
\modulecode{\infomodulecode}
\examiner{\infoexaminer}
\instructions{\infoinstructions}
\materials{\infomaterials}
\additionalinformation{\infoadditionalinformation}

\usepackage{longtable,booktabs,tabto,multicol}
\usepackage{amsmath,amssymb}

\providecommand{\tightlist}
{\setlength{\itemsep}{0pt}\setlength{\parskip}{0pt}}

\usepackage[totalmarks]{examquestions}
\usepackage{listings}


%%
%% defining style for code blocks
%%
\usepackage{inconsolata}
\usepackage{fontspec}
\setmonofont[Scale=0.8]{inconsolata}

\lstset{%
  frame=l,
  basicstyle=\ttfamily\linespread{1.1}\fontsize{12}{13}\selectfont,
  commentstyle=\itshape,
  columns=fullflexible,
  breaklines=true,
  numbersep=10pt,
  numbers=left,  
  framerule=0.5pt,
  xleftmargin=-16pt,
  language=Python,
  xrightmargin=-24pt,
  numberstyle=\fontsize{11}{13}\selectfont\color{gray},
  keywordstyle=\color{blue},
  rulecolor=\color{gray},
  stringstyle=\color{red},
  commentstyle=\color{green},
  stringstyle=\color{red},
  resetmargins=true
}

%%
%% defining the `question` environment
%%

\newcounter{bbquestion}
\newenvironment{bbquestion}[1][]{\sbox\qmarks{\bfseries #1 marks}
\refstepcounter{bbquestion}\par\medskip\textbf{Q.\thebbquestion.}
\hfill\begin{minipage}[t]{0.92\textwidth}}{\hspace*{0em plus 1fill}
\makebox{\bfseries [\usebox{\qmarks}]}\end{minipage}
\vspace{\subquestionskip}}

\usepackage{tikz,pgfplots,environ}
\usetikzlibrary{matrix,chains,positioning,decorations.pathreplacing,arrows,backgrounds}

\usepackage{parskip,setspace,xspace}
\onehalfspacing

\usepackage{etoolbox}
%% \usepackage{ifxetex}
%% \usepackage{ifluatex}

\newcommand{\btVFill}{\vskip0pt plus 1filll}

\usepackage{nccmath}
\usepackage{empheq}
\usepackage[many]{tcolorbox}
\usepackage{ragged2e}
\justifying

\usepackage{subfig}
\usepackage{changepage}
\usepackage{booktabs}
\usepackage[scale=2]{ccicons}
\usepackage{verbatim}
\usepackage[percent]{overpic}
\usepackage[absolute,overlay]{textpos}

\usepackage{array,booktabs}
\usepackage{multirow}
\usepackage{xspace}

\usepackage{adjustbox}
\captionsetup[subfigure]{labelformat=empty}

\usepackage{fancyhdr}


%% for displaying a watermark of the version used.  this can be useful
%% if you have multiple verions the same test to give. For instance,
%% version A for the even seat numbers, version B for the odd ones.

\ifdefined\infoversion

\usepackage{eso-pic}
\makeatletter
\AddToShipoutPicture{%
    \setlength{\@tempdimb}{.95\paperwidth}%
    \setlength{\@tempdimc}{0\paperheight}%
    \setlength{\unitlength}{1pt}%
    \put(\strip@pt\@tempdimb,\strip@pt\@tempdimc){%
        \makebox(0,0){ \textcolor{gray}{\infoversion} }%
    }%
}
\makeatother
\fi



\usepackage{svg}

%% \newcommand{\omrSIZE}{}
%% \newcommand{\omrSKIP}{.7em}

\ifcsname ifsolutions\endcsname\else
  \expandafter\let\csname ifsolutions\expandafter\endcsname
                  \csname iffalse\endcsname
                  \fi

%% loading the MCQ glyphs for keying (A-Z, 0-9). These glyphs are
%% suitable for OCR.

\usepackage{omrglyphs}

\ifsolutions
\newcommand{\iO}{\refstepcounter{enumii}\expandafter\csname o\Alph{enumii}\endcsname}
\newcommand{\iX}{\refstepcounter{enumii}\expandafter\csname x\Alph{enumii}\endcsname}
\else
\def\iO{\refstepcounter{enumii}\expandafter\csname o\Alph{enumii}\endcsname}
\def\iX{\refstepcounter{enumii}\expandafter\csname o\Alph{enumii}\endcsname}
\fi

\newcommand{\Q}[1]{{\\[1em]{\bf\textit Q#1. }}}

\ifsolutions
     \newenvironment{answer}[1][]
     {\par\medskip 
      \noindent \color{red} }
     { \medskip }
\else
	\usepackage{environ}
	\NewEnviron{answer}{%
    	\par  \medskip   \medskip

	}
\fi



\makeatletter
\def\cleardoublepage{\clearpage\if@twoside \ifodd\c@page\else
  \hbox{}
  \vspace*{\fill}
  \begin{center}
    This page was intentionally left blank.
  \end{center}
  \vspace{\fill}
  \thispagestyle{empty}
  \newpage
  \if@twocolumn\hbox{}\newpage\fi\fi\fi}
\makeatother

%%
%% start of user-defined preamble
%%



%%
%% end of user-defined preamble
%%

\begin{document}
\pagestyle{plain}
\maketitle

\cleardoublepage
\vspace*{-4em}

\ifdefstring{\infomidterm}{true}
{
\begin{minipage}[t]{.6\linewidth}
Student Name: \tabto{3cm} \underline{\hspace{6cm}}\vspace{0.3cm}
Student Number: \tabto{3.5cm} \underline{\hspace{5.5cm}}
\end{minipage}
\begin{minipage}[t]{.4\linewidth}
  Mark your student number below

  \omrDIGITS

  \omrDIGITS

  \omrDIGITS

  \omrDIGITS

  \omrDIGITS
      
  \omrDIGITS
      
  \omrDIGITS
      
  \omrDIGITS
      
\end{minipage}
}{
\begin{minipage}[t]{.6\linewidth}
Exam Number: \tabto{3cm} \underline{\hspace{6cm}}

Seat Number : \tabto{3.5cm} \underline{\hspace{5.5cm}}
\end{minipage}
\begin{minipage}[t]{.4\linewidth}
  Mark your exam number below

  \omrDIGITS

  \omrDIGITS

  \omrDIGITS

  \omrDIGITS

  \omrDIGITS
      
\end{minipage}
}

\vspace{1em}
{\bf 
  All your MCQ answers must be filled in on this answer page. 
}

For {\bf True} or {\bf False} questions, mark \oT or \oF. For
questions with multiple choices, mark all solutions that are correct
(for instance \xA\xB\oC\xD).

\vspace{0.5em}

\begin{multicols}{3}
\begin{enumerate}


\item[{\bf Q.1}]\oA\oB\oC\ifsolutions\xD\else\oD\fi


\item[{\bf Q.2}]\ifsolutions\oT\xF\else\oT\oF\fi

\end{enumerate}
\end{multicols}

\cleardoublepage
\ifsolutions\else
\renewcommand{\labelenumii}{(\alph{enumii})}
\renewcommand{\labelenumi}{(\alph{enumi})}
\fi



\begin{bbquestion}[5.0]
If vector ${\bf w}$ is of dimension $3 \times 1$ and matrix ${\bf A}$ of
dimension $5 \times 3$, then what is the dimension of $\left({\bf w}^{\top}{\bf
A}^{\top}{\bf A}{\bf w}\right)^{\top}$?
  \vspace{1em}

\begin{enumerate}\ifsolutions\else\renewcommand{\theenumi}{(\alph{enumi})}\fi\setcounter{enumii}{0}\setlength\itemsep{0.25em}
\ifsolutions  
\item[\iO] $5\times 5$
\else
\item $5\times 5$
\fi
\ifsolutions  
\item[\iO] $3\times 3$
\else
\item $3\times 3$
\fi
\ifsolutions  
\item[\iO] $3\times 1$
\else
\item $3\times 1$
\fi
\ifsolutions  
\item[\iX] $1\times 1$
\else
\item $1\times 1$
\fi
  \end{enumerate}
  
\end{bbquestion}
  


\begin{bbquestion}[5.0]
From this graph of the loss during training, we can say that there is underfitting.

\includegraphics[width=30em]{figures/fig-1.pdf}
  \vspace{1em}
\begin{enumerate}\setlength\itemsep{0.25em}
\ifsolutions  
\item[\oT] True
    \item[\xF] False
\else
    \item[(t)] True
    \item[(f)] False
\fi
\end{enumerate}
\end{bbquestion}

  



\begin{center}[oOo]\end{center}


\end{document}